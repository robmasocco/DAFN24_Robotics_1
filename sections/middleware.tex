% Section 2 - Middleware in robotics
% Roberto Masocco <roberto.masocco@uniroma2.it>
% May 14, 2023

% ### Middleware in robotics ###
\section{Middleware in robotics}
\graphicspath{{figs/section2/}}

% --- What is middleware? ---
\begin{frame}{What is middleware?}
	\begin{figure}
		\centering
		\includegraphics[width=.64\textwidth]{softwarePyramid.png}
		\caption{Software organization in a general-purpose computer system.}
		\label{fig:swpyramid}
	\end{figure}
\end{frame}
\begin{frame}{What is middleware?}
	\begin{block}{Definition of middleware}
		\justifying
		The term \textbf{middleware} identifies a kind of software that offers common services and functionalities to applications in addition to what an operating system usually does.
	\end{block}
	\justifying
	Middlewares are usually implemented as \textbg{libraries} that application programmers can use via appropriate \textbg{APIs}.
\end{frame}

% --- Middleware in robotics ---
\begin{frame}{Middleware in robotics}
	\justifying
	New problems arising when developing software for modern autonomous systems:
	\begin{itemize}
		\item integration of \textbg{sophisticated hardware} (microcontrollers, hardware accelerators, SBCs);
		\item \textbg{software} organization and maintenance;
		\item \textbg{communication} (involves both hardware and software!);
		\item debugging and \textbg{testing}.
	\end{itemize}
	\begin{block}{}
		\centering
		Middlewares can help to tackle and solve each one!
	\end{block}
\end{frame}

% --- Data Distribution Service ---
\begin{frame}{Data Distribution Service}
	\begin{block}{Definition of DDS}
		A DDS is a \textbf{publish-subscribe middleware} that handles communications between \textbf{real-time} systems and software over the network.
	\end{block}
	DDSs are currently used in automotive, aerospace, military...\\
	Their implementations follow an \textbg{open standard} that defines:
	\begin{itemize}
		\item \textbg{serialization} and \textbg{deserialization} of data packets (\texttt{RTPS Wire Protocol});
		\item \textbg{security protocols} and cryptographic operations;
		\item enforcing of \textbg{Quality of Service} policies to organize transmissions (specifying things like \textbg{queue sizes}, \textbg{best-effort} or \textbg{reliable} transmissions...);
	\end{itemize}
\end{frame}
\begin{frame}{Data Distribution Service}
	\begin{itemize}
		\item automatic discovery of \textbg{DDS participants} (over \textbg{multicast-IP/UDP}) and transmission of data (over \textbg{unicast-IP/UDP}) (\texttt{Discovery Protocol}).
	\end{itemize}
	\begin{figure}
		\centering
		\includegraphics[scale=.32]{ddsDomain.jpg}
		\caption{Scheme of a DDS-based network (data space).}
		\label{fig:ddsdomain}
	\end{figure}
\end{frame}
\begin{frame}{Data Distribution Service}
	DDS participants can either \textbg{publish to} or \textbg{subscribe to} a \textbg{topic}.
	\begin{block}{Definition of DDS topic}
		A DDS topic is \textbf{uniquely identified} by three attributes:
		\begin{itemize}
			\item a \textbf{name}, \emph{i.e.}, a human-readable character string;
			\item an \textbf{interface}, \emph{i.e.}, a custom packet format that specifies what data is carried over it (\emph{e.g.}, strings, numbers, arrays...);
			\item a \textbf{QoS policy} that specifies how transmissions should be performed.
		\end{itemize}
	\end{block}
	\begin{block}{}
		\centering
		\textbf{Changing even only one of the above results in a completely different topic!}
	\end{block}
\end{frame}
