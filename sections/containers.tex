% Section 4 - Containers
% Roberto Masocco <roberto.masocco@uniroma2.it>
% May 14, 2023

% ### Containers ###
\section{Containers}
\graphicspath{{figs/section4/}}

% --- Why containers? ---
\begin{frame}{Why containers?}
	\begin{exampleblock}{Example: Packaging applications}
		Suppose you are ready to distribute your new application:
		\begin{itemize}
			\item you need to be sure that it is compatible with all the \textbf{platforms} you chose to support;
			\item you need to figure out a way to deal with \textbf{dependencies};
			\item you want to publish some kind of \textbf{self-contained}, easily-identifiable \textbf{package}.
		\end{itemize}
	\end{exampleblock}
\end{frame}
\begin{frame}{Why containers?}
	\begin{exampleblock}{Example: Isolating applications}
		Suppose you are deploying applications on a server:
		\begin{itemize}
			\item you want to define \textbf{resource quotas} and \textbf{permissions} for each;
			\item you want to be sure that each module has what it needs to operate, but \textbf{nothing more};
			\item you want to \textbf{isolate} each module for security reasons, in case something goes wrong.
		\end{itemize}
	\end{exampleblock}
\end{frame}
\begin{frame}{Why containers?}
	\begin{exampleblock}{Example: Replicating environments}
		Suppose you are developing applications for a specific system (maybe with a different architecture):
		\begin{itemize}
			\item you want to have a \textbf{software copy} of such system without having to carry it with you;
			\item you want to have all \textbf{libraries} and \textbf{dependencies} installed without tainting your own;
			\item you would like to \textbf{deploy} the entire installation with just a few commands.
		\end{itemize}
	\end{exampleblock}
\end{frame}
\begin{frame}{Why containers?}
	\begin{columns}
		\column{.5\textwidth}
		A possible solution to many of the previous situations could be a set of \textbg{virtual machines}.\\
		However, virtual machines are \textbg{slow}, hypervisors take up \textbg{system resources} and guest kernels must always be \textbg{tweaked}.
		\newline\newline
		In each of the above scenarios something simpler would be enough, especially since \textbg{the OS is not involved}, only applications are.
		\begin{block}{}
			\centering
			This is what a \textbf{container} is.
		\end{block}

		\column{.5\textwidth}
		\begin{figure}
			\centering
			\includegraphics[scale=.7]{freebsdjail.png}
			\label{fig:jail}
			\caption{FreeBSD jail logo.}
		\end{figure}
	\end{columns}
\end{frame}

% --- Containers in the Linux kernel ---
\begin{frame}{Containers in the Linux kernel}
	\begin{columns}
		\column{.5\textwidth}
		Support for containers was added to the Linux kernel with a set of \textbg{features} starting from kernel 2.6 (2003), mainly:
		\begin{itemize}
			\item \textbg{control groups} (\texttt{cgroups}): defining different resource usage policies for groups of processes;
			\item \textbg{namespaces}: isolating processes and users in different "realms", both hardware (e.g. network stack) and software (e.g. PIDs);
			\item \textbg{capabilities}: granting some of the superuser's permissions to unprivileged threads.
		\end{itemize}

		\column{.5\textwidth}
		\begin{figure}
			\centering
			\includegraphics[scale=.2]{tux.png}
			\label{fig:tux}
			\caption{Tux.}
		\end{figure}
	\end{columns}
\end{frame}
\begin{frame}{Containers in the Linux kernel}
	\begin{figure}
		\centering
		\includegraphics[scale=.137]{pidNamespace.png}
		\label{fig:pidnamespace}
		\caption{Nested PID namespaces.}
	\end{figure}
\end{frame}
