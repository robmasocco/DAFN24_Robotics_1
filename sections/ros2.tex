% Section 3 - ROS 2 overview
% Roberto Masocco <roberto.masocco@uniroma2.it>
% March 13, 2022

% ### ROS 2 overview ###
\section{ROS 2 overview}
\graphicspath{{figs/section3/}}

% --- What is ROS 2? ---
\begin{frame}{What is ROS 2?}
	\begin{columns}
		\column{.5\textwidth}
		\only<1>{
			\nolistindent ROS 2 is a \textbg{DDS-based}, \textbg{open-source} middleware for robotic applications.\\
			It allows developers to build and manage \textbg{distributed control architectures} made of many modules, usually referred to as \textbg{nodes}.
		}
		\only<2>{
			\nolistindent \nolistindent ROS 2 currently supports \textbg{C++} and \textbg{Python} for application programming, and runs natively on \textbg{Ubuntu Linux 22.04}.
			\newline\newline
			New versions are periodically released as \textbg{distributions}: the current LTS one is \textbg{Humble Hawksbill}; the development version is \textbg{Rolling Ridley} and can only be compiled from source.\\
			It is available as \textbg{binary \texttt{deb} packages} for \texttt{x86} and \texttt{ARMv8} architectures.
			\newline\newline
			A \textbg{distribution} is a collection of software packages: \textbg{libraries}, \textbg{tools}, and \textbg{applications}.
		}

		\column{.5\textwidth}
		\begin{figure}
			\centering
			\includegraphics[width=.8\textwidth]{ros2Logo.jpg}
			\caption{ROS 2 logo.}
			\label{fig:ros2logos}
		\end{figure}
	\end{columns}
\end{frame}

% --- Why ROS 2? ---
\begin{frame}{Why ROS 2?}
	\begin{columns}
		\column{.5\textwidth}
		The ROS project started in 2007, to provide a middleware that could solve the \textbg{software integration} and \textbg{communication} problems in robotics. It has evolved much since then.
		\newline\newline
		ROS 2 helps to design and build \textbg{distributed control architectures}, providing a common ground for the \textbg{integration} of different systems, sensors, actuators, and algorithms.\\
		It is a common framework for the development of \textbg{robotics software}.
		\newline\newline
		Its adoption is still limited because of familiarity with the original ROS, but it is \textbg{growing}.

		\column{.5\textwidth}
		\begin{figure}
			\centering
			\includegraphics[scale=.17]{why_ros2.png}
			\label{fig:whyros2}
			\caption{STM32 (bottom), Raspberry Pi (middle), and Nvidia Xavier AGX (top).}
		\end{figure}
	\end{columns}
\end{frame}

% --- Main Features ---
\begin{frame}{Main features}
	As a middleware, it offers many \textbg{services to roboticists}, including:
	\begin{itemize}
		\item \textbg{three communication paradigms}, easy to set up and based on the DDS: \textbg{messages}, \textbg{services} and \textbg{actions};
		\item organization of software packages, allowing for \textbg{redistribution and code reuse}, thanks to the \textbg{\texttt{colcon}} package manager;
		\item module configuration tools: \textbg{node parameters} and \textbg{launch files};
		\item integrated \textbg{logging subsystem} (involves both console and log files);
		\item CLI \textbg{introspection tools} for debugging and testing;
	\end{itemize}
\end{frame}
\begin{frame}{Main features}
	\begin{itemize}
		\item may be integrated in some \textbg{simulators} (\emph{e.g.}, Gazebo) and \textbg{visualizers} (\emph{e.g.}, RViz).
	\end{itemize}
	\begin{figure}
		\centering
		\includegraphics[scale=.133]{simulation.png}
		\label{fig:sim}
		\caption{Simulated drone in Gazebo Classic and RViz.}
	\end{figure}
\end{frame}

% --- Application Programming Interface ---
\begin{frame}{Application Programming Interface}{Deep dive into ROS 2 internals}
	A \textbg{ROS 2 installation}, from bottom to top, works as follows:
	\begin{enumerate}
		\item \textbg{DDS}: the middleware that implements the \textbg{communication layer} (many different implementations are supported).
		\item \textbg{RMW}: ROS MiddleWare, is the \textbg{DDS abstraction layer}, which allows to use different DDS implementations without changing the application code.
		\item \textbg{rcl}: ROS Client Support Library (C), implements basic entities: \textbg{nodes}, \textbg{publishers}, \textbg{subscribers}, \textbg{services}, \textbg{clients}, and \textbg{timers}.
		\item \textbg{rclc}/\textbg{rclcpp}/\textbg{rclpy}: ROS Client Library (C/C++/Python), implements the same entities as \texttt{rcl}, plus extended functionalities like the \textbg{executor}: a job scheduler.
	\end{enumerate}
	Then, there are \textbg{packages}: libraries, tools, and applications, both officially provided and community-contributed. The entire ROS 2 codebase is on \href{https://github.com/ros2}{\color{blue}\underline{GitHub}}.\\
  Keep this in mind during debugging, or while looking for information on an API!
\end{frame}
\begin{frame}{Application Programming Interface}{How a ROS 2 application works}
	The most important entity is the \textbg{node}, whose functionalities are specified upon creation.\\
	A node must usually do \textbg{a single thing}, being the \textbg{unit} in a \textbg{distributed architecture}.\\
	With its class methods, it can:
	\begin{itemize}
		\item act as an \textbg{entry point} towards the DDS layer, to handle communications;
		\item embed \textbg{software modules} like data, algorithms, and threads, that implement the application logic;
		\item register \textbg{callbacks} to handle \textbg{events}, such as timers or incoming messages.
	\end{itemize}
	Thus, it is both an \textbg{operational unit} and a \textbg{communication endpoint}.\\
	Nodes are usually handled by \textbg{executors}, which are responsible for scheduling and processing their \textbg{ROS-workload}.
	\begin{block}{}
		\centering
		\textbf{Nodes are just objects in your application: they can embed any kind of software module, but they do not limit the design to their paradigms.}
	\end{block}
\end{frame}

% --- Executors: events and callbacks ---
\begin{frame}{Executors}{Handling events and callbacks}
	\begin{figure}
		\centering
		\includesvg[scale=.45]{ROS2_executor_scheme.svg}
		\label{fig:executorscheme}
		\caption{ROS 2 event-based programming paradigm.}
	\end{figure}
\end{frame}
\begin{frame}{Executors}{Handling events and callbacks}
	\begin{enumerate}
		\item Middleware functionalities trigger \textbg{(a)synchronous events}.
		\item Events are handled by \textbg{background jobs}, coded in \textbg{callbacks} by the programmer.
		\item Callbacks are \textbg{registered} into a \textbg{node} when its functionalities are specified (\emph{e.g.}, upon creation).
		\item The workload that a node carries is scheduled and processed by an \textbg{executor}, single- or multi-threaded.
	\end{enumerate}
	\begin{alertblock}{}
		\centering
		Executors implement a \textbr{round-robin}, \textbr{non-preemptive} policy that \textbr{always prioritizes timers}.
	\end{alertblock}
\end{frame}

% --- Flaws ---
\begin{frame}{Flaws}
	\visible<1->{
		\begin{alertblock}{ROS 2 biggest flaws (as of today)}
			The main concerns arise when developing low-level stuff:
			\begin{itemize}
				\item \textbr{the DDS layer is almost completely abstracted}, so non-standard network configurations may get tricky;
				\item the internal job scheduling algorithm (namely the \textbr{executor}) is \textbr{not suited for hard real-time applications} because of its \textbr{non-preemptive} nature.
			\end{itemize}
		\end{alertblock}
	}
	\visible<2>{
		What to do when development gets to a really low level?
		\begin{itemize}
			\item Hand off stuff to dedicated \textbg{microcontrollers}.
			\item Use \textbg{micro-ROS}: hard real-time ROS 2 on microcontrollers and different communication interfaces.
		\end{itemize}
	}
\end{frame}
